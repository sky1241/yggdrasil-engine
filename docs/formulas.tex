\documentclass[11pt,a4paper]{article}
\usepackage[utf8]{inputenc}
\usepackage[T1]{fontenc}
\usepackage[french]{babel}
\usepackage{amsmath,amssymb,amsfonts}
\usepackage{hyperref}
\usepackage{booktabs}
\usepackage{geometry}
\geometry{margin=2.5cm}

\title{\textbf{Yggdrasil Engine --- Formules} \\[0.5em]
\large Sources, modifications et mapping vers le mycelium}
\author{Sky \& Claude (Opus 4.6) --- Session 6, 23 f\'evrier 2026}
\date{}

\begin{document}
\maketitle

% ============================================================================
\section{Architecture des strates}
% ============================================================================

\begin{center}
\begin{tabular}{clll}
\toprule
\textbf{Strate} & \textbf{Nom} & \textbf{Contenu} & \textbf{Mycelium?} \\
\midrule
S6 & Ind\'ecidable & G\"odel, Halting problem & Non \\
S5 & Presque ind\'ecidable & & Non \\
S4 & Logique sup\'erieure & & Non \\
S3 & Conjectures & Riemann, P=NP & Non \\
S2 & R\'ecursion\textsuperscript{2} & & Non \\
S1 & Structures r\'ecursives & & Non \\
\midrule
\textbf{S0} & \textbf{Formules prouv\'ees} & \textbf{21,524 symboles C1} & \textbf{Oui} \\
S-1 & M\'etiers & physics, biology, eng. & Oui \\
S-2 & Glyphes & $=$, $+$, $\int$, $\Sigma$, $\partial$ & Oui \\
\bottomrule
\end{tabular}
\end{center}

Le mycelium (r\'eseau de co-occurrences) vit dans le \textbf{sol} (S-2 \`a S0).
Au-dessus de S0 = l'arbre (conjectures, abstractions). Pas de mycelium.

% ============================================================================
\section{Formules utilis\'ees --- Sources originales}
% ============================================================================

% ----------------------------------------------------------------------------
\subsection{Fitness $\eta_i$ --- Wang, Song \& Barab\'asi (2013)}
% ----------------------------------------------------------------------------

\textbf{Source:} Wang, D., Song, C., \& Barab\'asi, A.-L. (2013).
\textit{Quantifying Long-term Scientific Impact}.
Science, 342(6154), 127--132.
\href{https://doi.org/10.1126/science.1237825}{doi:10.1126/science.1237825}

\textbf{Formule originale:}
\begin{equation}
c_i(t) = \eta_i \cdot m \cdot \sum_j (c_j + 1) \cdot P_i(t - t_i)
\end{equation}
o\`u $P_i(t)$ est un vieillissement log-normal:
\begin{equation}
P_i(t) = \frac{1}{t} \exp\!\left(-\frac{(\ln t - \mu_i)^2}{2\sigma_i^2}\right)
\end{equation}

\textbf{Adaptation Yggdrasil} (\texttt{engine/core/scisci.py:15}):
\begin{equation}
\eta_i \approx \frac{\text{citations\_totales}}{\sum_{t=1}^{\text{age}} P_i(t)}
\quad\text{avec } \mu=1, \sigma=1
\end{equation}
Simplification: on estime $\eta_i$ comme ratio citations r\'eelles / citations attendues
selon l'\^age, sans le terme de pr\'ef\'erence cumulative.

% ----------------------------------------------------------------------------
\subsection{D-index (Disruption) --- Wu, Wang \& Evans (2019)}
% ----------------------------------------------------------------------------

\textbf{Source:} Wu, L., Wang, D., \& Evans, J. A. (2019).
\textit{Large Teams Develop and Small Teams Disrupt Science and Technology}.
Nature, 566, 378--382.
\href{https://doi.org/10.1038/s41586-019-0941-9}{doi:10.1038/s41586-019-0941-9}

Formule originale (Funk \& Owen-Smith, 2017):
\begin{equation}
D = \frac{n_i - n_j}{n_i + n_j + n_k}
\end{equation}
\begin{itemize}
\item $n_i$ = papers citant le focal MAIS PAS ses r\'ef\'erences
\item $n_j$ = papers citant le focal ET ses r\'ef\'erences
\item $n_k$ = papers citant SEULEMENT les r\'ef\'erences
\end{itemize}
$D \to +1$: disruptif (nouveau paradigme).
$D \to -1$: developmental (extension).

\textbf{Adaptation Yggdrasil} (\texttt{engine/core/scisci.py:55}):
Utilis\'e tel quel. Sert \`a d\'etecter les trous techniques (type A: $|D|$ bas)
et perceptuels (type C: $|D|$ haut mais citations basses).

% ----------------------------------------------------------------------------
\subsection{Z-score d'atypicit\'e --- Uzzi et al. (2013)}
% ----------------------------------------------------------------------------

\textbf{Source:} Uzzi, B., Mukherjee, S., Stringer, M., \& Jones, B. (2013).
\textit{Atypical Combinations and Scientific Impact}.
Science, 342(6157), 468--472.
\href{https://doi.org/10.1126/science.1240474}{doi:10.1126/science.1240474}

\begin{equation}
z = \frac{f_{\text{obs}} - \mu_{\text{random}}}{\sigma_{\text{random}}}
\end{equation}
$z \ll 0$: combinaison tr\`es atypique (novel).
$z \gg 0$: combinaison conventionnelle.

\textbf{Adaptation Yggdrasil} (\texttt{engine/core/scisci.py:83}):
Utilis\'e tel quel. Sert au score de trou conceptuel (type B):
plus le z-score est n\'egatif, plus la paire est un trou potentiel.

% ----------------------------------------------------------------------------
\subsection{Q-factor --- Sinatra et al. (2016)}
% ----------------------------------------------------------------------------

\textbf{Source:} Sinatra, R., Wang, D., Deville, P., Song, C., \& Barab\'asi, A.-L. (2016).
\textit{Quantifying the Evolution of Individual Scientific Impact}.
Science, 354(6312), aaf5239.
\href{https://doi.org/10.1126/science.aaf5239}{doi:10.1126/science.aaf5239}

\begin{equation}
\log(c_i) = \log(Q_i) + \log(p_i)
\end{equation}
$Q$ = qualit\'e constante d'un chercheur. $p$ = chance (random).

\textbf{Adaptation Yggdrasil} (\texttt{engine/core/scisci.py:107}):
\begin{equation}
Q \approx \text{median}\!\left(\log(c_i + 1)\right)
\end{equation}
Simplification: m\'ediane des log-citations comme proxy de qualit\'e.

% ============================================================================
\section{Formules de d\'etection de trous}
% ============================================================================

\textbf{Fichier:} \texttt{engine/core/holes.py}

% ----------------------------------------------------------------------------
\subsection{Type A --- Trou Technique}
% ----------------------------------------------------------------------------

Tout le monde SAIT o\`u aller, personne ne PEUT.
Exemples: Poincar\'e (Hamilton bloqu\'e 20 ans), Higgs (besoin du LHC).

\begin{equation}
\text{Score}_A = \text{Production}(t) \times (1 - |\Delta\eta / \Delta t|) \times (1 - |D|)
\end{equation}
Production \'elev\'ee $\times$ fitness stagnante $\times$ D-index bas (developmental).

% ----------------------------------------------------------------------------
\subsection{Type B --- Trou Conceptuel}
% ----------------------------------------------------------------------------

Personne n'a l'ID\'EE de connecter. Le vide est invisible.
Exemples: GANs (game theory $\times$ deep learning), CRISPR, AlphaFold.

\begin{equation}
\text{Score}_B = \text{Activity}(A) \times \text{Activity}(B) \times (1 - \text{CoOcc}(A,B)) \times \frac{|z|}{10}
\end{equation}

% ----------------------------------------------------------------------------
\subsection{Type C --- Trou Perceptuel}
% ----------------------------------------------------------------------------

L'outil EXISTE, personne n'y CROIT.
Exemples: mRNA (Karik\'o rejet\'ee 30 ans), H.\ pylori, quasicristaux.

\begin{equation}
\text{Score}_C = \eta_i \times |D| \times \left(1 - \frac{c_i(t)}{c_i^{\text{expected}}}\right)
\end{equation}

% ============================================================================
\section{Topologie du mycelium}
% ============================================================================

\textbf{Fichier:} \texttt{engine/pipeline/mycelium\_full.py} (24 briques, 456 tests)

% ----------------------------------------------------------------------------
\subsection{Meshedness $\alpha$ (brique 1)}
% ----------------------------------------------------------------------------

\textbf{Source:} Bebber, D. P., Hynes, J., Darrah, P. R., Boddy, L., \& Fricker, M. D. (2007).
\textit{Biological solutions to transport network design}.
Proc.\ R.\ Soc.\ B, 274, 2307--2315.

\begin{equation}
\alpha = \frac{L - N + 1}{2N - 5}
\end{equation}
$L$ = ar\^etes, $N$ = n\oe uds. $\alpha = 0$: arbre pur. $\alpha = 1$: r\'eseau planaire maximal.

\textbf{Adaptation Yggdrasil:} utilis\'e tel quel. Le $\alpha$ mesure la r\'esilience
du mycelium dans le sol (S-2 \`a S0). Sert au delta $\alpha$ dans les bougies OHLC.

% ----------------------------------------------------------------------------
\subsection{Betweenness Centrality (brique 5)}
% ----------------------------------------------------------------------------

\textbf{Source:} Freeman, L. C. (1977).
\textit{A Set of Measures of Centrality Based on Betweenness}.
Sociometry, 40(1), 35--41.

\begin{equation}
BC(v) = \sum_{s \neq v \neq t} \frac{\sigma_{st}(v)}{\sigma_{st}}
\end{equation}
$\sigma_{st}$ = chemins les plus courts entre $s$ et $t$.
$\sigma_{st}(v)$ = ceux passant par $v$.

\textbf{Adaptation Yggdrasil:} utilis\'e via NetworkX (normalis\'e). Un n\oe ud \`a BC \'elev\'e
est un pont (P1). Corr\`ele avec le flux r\'eel en mycelium (Oyarte Galvez 2025).

% ----------------------------------------------------------------------------
\subsection{Physarum (brique 10)}
% ----------------------------------------------------------------------------

\textbf{Source:} Tero, A., Takagi, S., Saigusa, T., et al. (2010).
\textit{Rules for biologically inspired adaptive network design}.
Science, 327(5964), 439--442.
\href{https://doi.org/10.1126/science.1177894}{doi:10.1126/science.1177894}

\begin{equation}
\frac{dD_{ij}}{dt} = f(|Q_{ij}|) - \gamma D_{ij}
\end{equation}
$D_{ij}$ = conductivit\'e de l'hyphe. $Q_{ij}$ = flux. $\gamma$ = d\'ecroissance.
$f(|Q|) = |Q|^\mu$ avec $\mu = 1$ (shortest path) ou $\mu < 1$ (loops tolerated).

\textbf{Adaptation Yggdrasil:} les hyphes vivantes/mortes du Physarum servent
au delta 6 (redistribution des flux) dans les bougies OHLC.

% ----------------------------------------------------------------------------
\subsection{Laplacien spectral (positionnement)}
% ----------------------------------------------------------------------------

\textbf{Source:} Chung, F. R. K. (1997).
\textit{Spectral Graph Theory}. AMS.

Laplacien normalis\'e:
\begin{equation}
\mathcal{L} = D^{-1/2} L D^{-1/2} = I - D^{-1/2} A D^{-1/2}
\end{equation}
$A$ = matrice de co-occurrence (log-transform\'ee). $D$ = matrice des degr\'es.

Vecteur de Fiedler (2\`eme plus petite valeur propre):
\begin{equation}
\mathcal{L} \mathbf{v}_2 = \lambda_2 \mathbf{v}_2
\end{equation}

\textbf{Adaptation Yggdrasil} (\texttt{engine/topology/cooccurrence\_scan.py:225}):
\begin{equation}
\text{position}_x = v_2[i], \quad \text{position}_z = v_3[i]
\end{equation}
Les 2\`eme et 3\`eme vecteurs propres donnent les coordonn\'ees spectrales.
Appliqu\'e sur la matrice de co-occurrence \textbf{log-transform\'ee}: $W = \log(1 + C)$
o\`u $C$ est la matrice de co-occurrence brute.

Positionne TOUT le sol: glyphes (S-2), m\'etiers (S-1), formules (S0).

% ============================================================================
\section{Formules Sedov-Taylor --- Impact m\'et\'eorite}
% ============================================================================

% ----------------------------------------------------------------------------
\subsection{Formules originales (physique)}
% ----------------------------------------------------------------------------

\textbf{Sources:}
\begin{itemize}
\item Taylor, G. I. (1950). \textit{The Formation of a Blast Wave by a Very Intense Explosion}.
Proc.\ R.\ Soc.\ A, 201(1065), 159--174.
\item Sedov, L. I. (1946). \textit{Propagation of strong shock waves}.
J.\ Appl.\ Math.\ Mech., 10, 241--250.
\item von Neumann, J. (1947). \textit{The point source solution}. In: Blast Wave (report).
\end{itemize}

\textbf{Rayon du front de choc:}
\begin{equation}
\boxed{R(t) = \beta \left(\frac{E t^2}{\rho_0}\right)^{1/5}}
\end{equation}
\begin{itemize}
\item $E$ = \'energie totale inject\'ee (J)
\item $\rho_0$ = densit\'e du milieu ambiant (kg/m$^3$)
\item $t$ = temps depuis l'explosion
\item $\beta$ = constante sans dimension:
  $\beta \approx 1.033$ pour $\gamma = 7/5$ (air),
  $\beta \approx 1.152$ pour $\gamma = 5/3$ (gaz monoatomique)
\end{itemize}

\textbf{Vitesse du choc:}
\begin{equation}
\dot{R}(t) = \frac{2}{5} \beta \left(\frac{E}{\rho_0}\right)^{1/5} t^{-3/5}
\end{equation}
Le choc d\'ec\'el\`ere en $t^{-3/5}$.

\textbf{Conditions post-choc (Rankine-Hugoniot, \`a $\xi = 1$):}
\begin{align}
V(1) &= \frac{2}{\gamma + 1} \\
G(1) &= \frac{\gamma + 1}{\gamma - 1} \\
Z(1) &= \frac{2}{\gamma + 1}
\end{align}
Pour $\gamma = 7/5$: densit\'e post-choc $= 6\times$ la densit\'e ambiante.

\textbf{Conservation d'\'energie:}
\begin{equation}
E = 4\pi \rho_0 R^3 \dot{R}^2 \left[
2\int_0^1 G(\xi) V^2(\xi) \xi^2 \, d\xi
+ \frac{2}{\gamma - 1} \int_0^1 Z(\xi) \xi^2 \, d\xi
\right]
\end{equation}
Partition pour $\gamma = 7/5$: $\sim$30\% cin\'etique, $\sim$70\% thermique.

% ----------------------------------------------------------------------------
\subsection{Scaling de crat\`ere (Holsapple-Schmidt)}
% ----------------------------------------------------------------------------

\textbf{Source:} Holsapple, K. A. (1993).
\textit{The Scaling of Impact Processes in Planetary Sciences}.
Annu.\ Rev.\ Earth Planet.\ Sci., 21, 333--373.

Groupes $\pi$ adimensionnels:
\begin{align}
\pi_2 &= \frac{g \cdot a}{U^2} & \text{(gravit\'e vs inertie)} \\
\pi_3 &= \frac{Y}{\rho_t U^2} & \text{(r\'esistance vs inertie)} \\
\pi_4 &= \frac{\rho_t}{\rho_p} & \text{(densit\'es cible/projectile)} \\
\pi_V &= \frac{\rho_t V}{m} & \text{(volume crat\`ere adimensionnel)}
\end{align}

Loi de scaling compl\`ete:
\begin{equation}
\pi_V = K_1 \left[
\pi_2 \cdot \pi_4^{(6\nu - 2 - \mu)/(3\mu)}
+ K_2 \cdot \pi_3 \cdot \pi_4^{(6\nu - 2)/(3\mu)}
\right]^{-3\mu/(2+\mu)}
\end{equation}

R\'egime gravit\'e ($\pi_2 \gg \pi_3$, gros impacts):
\begin{equation}
\pi_V = K_1 \cdot \pi_2^{-3\mu/(2+\mu)} \cdot \pi_4^{(6\nu-2-\mu)/(2+\mu)}
\end{equation}

R\'egime r\'esistance ($\pi_3 \gg \pi_2$, petits impacts):
\begin{equation}
\pi_V = K_1 K_2^{-3\mu/(2+\mu)} \cdot \pi_3^{-3\mu/(2+\mu)} \cdot \pi_4^{(6\nu-2)/(2+\mu)}
\end{equation}

Exposant $\mu$: 2/3 (scaling \'energie) \`a 1/3 (scaling quantit\'e de mouvement).
Valeur exp\'erimentale pour mat\'eriaux consolid\'es: $\mu \approx 0.58$.

% ----------------------------------------------------------------------------
\subsection{Mapping Sedov-Taylor $\to$ Mycelium Yggdrasil}
% ----------------------------------------------------------------------------

\begin{center}
\begin{tabular}{lll}
\toprule
\textbf{Physique} & \textbf{Yggdrasil} & \textbf{Mesure} \\
\midrule
$E$ (\'energie) & strate\_height $\times$ continents\_touch\'es & entier \\
$\rho_0$ (densit\'e) & meshedness $\alpha$ locale avant impact & $[0, 1]$ \\
$R(t)$ (rayon choc) & concepts affect\'es \`a $t$ mois & entier \\
$\dot{R}(t)$ (vitesse) & nouvelles co-occurrences / mois & taux \\
$\gamma$ (adiabatique) & rigidit\'e du sol local & \`a calibrer \\
$\beta$ (constante) & \`a calibrer depuis bo\^ites de m\'et\'eorites & $\sim 1$ \\
\bottomrule
\end{tabular}
\end{center}

\textbf{Formule adapt\'ee:}
\begin{equation}
\boxed{R(t) = \beta_{\text{myc}} \left(\frac{E_{\text{impact}} \cdot t^2}{\alpha_{\text{local}}}\right)^{1/5}}
\end{equation}
avec:
\begin{align}
E_{\text{impact}} &= \text{strate\_height} \times \text{continents\_touch\'es} \\
\alpha_{\text{local}} &= \text{meshedness avant impact (densit\'e du sol S-2 \`a S0)}
\end{align}

\textbf{Pr\'ediction testable (corr\'elation bougie $\leftrightarrow$ type de trou):}
\begin{itemize}
\item Trou A (technique): $\alpha_{\text{local}}$ \'elev\'e $\to$ blast lent $\to$ bougie moyenne
\item Trou B (conceptuel): $\alpha_{\text{local}}$ faible $\to$ blast rapide $\to$ bougie courte
\item Trou C (perceptuel): $\alpha_{\text{local}}$ \'elev\'e + hostile $\to$ blast bloqu\'e $\to$ bougie longue
\end{itemize}

\textbf{Calibration:} depuis Shannon 1948 (premi\`ere bo\^ite, $\rho_0$ mesurable).
Test final: G\"odel 1931 (avant lui tout est plat, $\rho_0 \approx 0$).

% ============================================================================
\section{Bougies OHLC et 7 deltas}
% ============================================================================

\textbf{Fichier pr\'evu:} \texttt{engine/meteorites.py} (V3, apr\`es le scan)

Chaque perc\'ee majeure = une bougie OHLC:
\begin{center}
\begin{tabular}{ll}
\toprule
\textbf{OHLC} & \textbf{D\'efinition} \\
\midrule
Open & Date d'\'emission du paper \\
High & Pic de reconfiguration maximale du mycelium \\
Low & Creux (r\'esistance paradigme / stabilisation) \\
Close & Date de validation (prouv\'e, r\'epliqu\'e, explosion citations) \\
\bottomrule
\end{tabular}
\end{center}

Longueur de bougie $= t_{\text{Close}} - t_{\text{Open}}$ = temps de r\'esistance du paradigme.

\textbf{7 indicateurs techniques} (deltas entre frame avant et apr\`es):
\begin{align}
\Delta_1 &= \text{volume} = \text{nouvelles ar\^etes co-occurrence cr\'e\'ees} \\
\Delta_2 &= \text{amplitude} = \|\mathbf{x}_{\text{apr\`es}} - \mathbf{x}_{\text{avant}}\|_2
  \quad\text{(d\'eplacement spectral des centro\"\i des)} \\
\Delta_3 &= \Delta BC = BC_{\text{apr\`es}} - BC_{\text{avant}}
  \quad\text{(nouveaux ponts vs obsol\`etes)} \\
\Delta_4 &= \Delta\alpha = \alpha_{\text{apr\`es}} - \alpha_{\text{avant}}
  \quad\text{(r\'esilience r\'eseau)} \\
\Delta_5 &= \Delta P4 = P4_{\text{ferm\'es}} - P4_{\text{ouverts}}
  \quad\text{(trous net)} \\
\Delta_6 &= \Delta\text{Physarum} = \text{hyphes cr\'e\'ees} - \text{hyphes mortes} \\
\Delta_7 &= \text{births} = \text{concepts n\'es} - \text{concepts morts}
\end{align}

\textbf{Impact pond\'er\'e:}
\begin{equation}
I_{\text{m\'et\'eorite}} = \text{strate\_height} \times \text{continents\_touch\'es}
\end{equation}
Exemples: Poincar\'e ($S3 \times 1 = 3$), Shannon ($S1 \times 7 = 7$),
G\"odel ($S6 \times 9 = 54$, hors \'echelle).

% ============================================================================
\section{Co-occurrence et force de lien}
% ============================================================================

\textbf{Fichier:} \texttt{engine/core/scisci.py:128}

\begin{equation}
\text{CoOcc}(A,B) = \frac{P(A \cap B)}{P(A) \cdot P(B) \cdot N}
= \frac{n_{AB}}{(n_A / N) \cdot (n_B / N) \cdot N}
\end{equation}
$> 1$: sur-repr\'esent\'e. $< 1$: sous-repr\'esent\'e. $= 0$: jamais vu ensemble (trou P4).

\textbf{Jaccard modifi\'e:}
\begin{equation}
J(A,B) = \frac{n_{AB}}{n_A + n_B - n_{AB}}
\end{equation}

% ============================================================================
\section{Hypoth\`eses \`a v\'erifier (V3)}
% ============================================================================

\begin{enumerate}
\item \textbf{Partition \'energ\'etique:} $\sim$30\% nouveaux ponts / $\sim$70\% renforcement?
  (par analogie avec 30\% cin\'etique / 70\% thermique pour $\gamma = 7/5$).
  Mesurer avec $\Delta BC$ vs $\Delta\alpha$.
\item \textbf{$\gamma$ variable:} le $\gamma$ effectif du mycelium varie-t-il par r\'egion?
  (sol dense = $\gamma$ \'elev\'e, sol creux = $\gamma$ bas).
  Calibrer depuis les bo\^ites de m\'et\'eorites.
\item \textbf{Exposant 2/5:} le rayon $R(t) \propto t^{2/5}$ tient-il?
  Fitter la courbe sur les perc\'ees post-1948 et mesurer l'exposant r\'eel.
\item \textbf{Corr\'elation bougie $\leftrightarrow$ $\rho_0$:}
  les bougies longues correspondent-elles aux zones denses (trou C)?
\end{enumerate}

\end{document}
